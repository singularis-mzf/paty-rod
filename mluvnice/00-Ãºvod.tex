% Pátý rod
% Copyright (c) 2021 Singularis <singularis@volny.cz>
%
% Toto dílo je dílem svobodné kultury; můžete ho šířit a modifikovat pod
% podmínkami licence Creative Commons Attribution-ShareAlike 4.0 International
% vydané neziskovou organizací Creative Commons. Text licence je přiložený
% k tomuto projektu nebo ho můžete najít na webové adrese:
%
% https://creativecommons.org/licenses/by-sa/4.0/
%
\documentclass[10pt,draft]{article}
\usepackage[english,czech]{babel}
\usepackage[xetex,
    layout=a4paper,inner=1.5cm,outer=1cm,top=2cm,bottom=1.5cm,
    twoside,
    layouthoffset=0mm,layoutvoffset=0mm,
    a4paper
    ]{geometry}
\usepackage[xetex,table]{xcolor}  % barevné písmo
\usepackage{changepage}     % prostředí {adjustwidth}
%\usepackage{colortbl}       % barevné pozadí prostředí {tabular}
\usepackage{enumitem}       % nastavení {enumerable} a {itemize}
\usepackage{etoolbox}       % podmíněný překlad apod.
\usepackage{fancyhdr}       % formátování záhlaví a zápatí
\usepackage{fontspec}       % {fontspec} musí být načten až po balíčcích s matematickými symboly!
\usepackage[bottom]{footmisc}% aby se poznámky pod čarou správně umístily
\usepackage{ifthen}         %
\usepackage{longtable}       % vyžadována pro {tabu}
%\usepackage[xetex]{graphicx}% vkládání obrázků
%\usepackage{multicol}       % prostředí {multicols} a {multicols*}
\usepackage{tabu}           % lepší tabulky (prostředí {tabu})
\usepackage{titlesec}       % formátování nadpisů kapitol
%\usepackage[titles]{tocloft}% formátování přehledu („obsahu“)
\usepackage{verbatim}

\newboolean{gm}\setboolean{gm}{true}%
\newcommand{\ifgmelse}[2]{\ifthenelse{\boolean{gm}}{#1}{#2}}

% {fontspec}:
    \setmainfont{Latin Modern Roman}\relax%
    \setsansfont{Latin Modern Sans}\relax%
    \setmonofont{Latin Modern Mono Light}\relax%

    \definecolor{bila}{gray}{1}
    \definecolor{cerna}{gray}{0}
    \definecolor{svetleseda}{gray}{0.9}
    \definecolor{seda}{gray}{0.5}
\begin{document}\rmfamily%
%
% TITULNÍ STRANA
%
%\pagestyle{empty}%
\begin{center}%
{\Huge\bfseries PÁTÝ ROD\\}\medskip%
Návrh podle: Singularis\\%
Verze: 1.1 (leden 2022)
\end{center}%
\bigskip%
%
\section{Úvod (deklarace)}

Pátý rod je umělé rozšíření (nadstavba) českého jazyka,
jehož cílem je doplnit češtině prostředky pro pohlavně/genderově neutrální
vyjadřování v hovoru i textu, které bude esteticky a funkčně srovnatelné
s rodem ženským a rodem mužským životným.
\ifgmelse{%
    Pátý rod je vhodný zejména jako alternativa generického maskulina
    (jež v mnoha kontextech není vnímáno pohlavně/genderově neutrálně
    -- viz např. slovo „striptér“) a dosud používaných lomítkových tvarů
    (např. „lingvista/ka“).}

Pátý rod nabízí nové názvy osob (např. „studenx“, „mažorex“, „striptéřix“ či „prostitux“),
které příponou „-x“ výslovně odmítají vyjádření pohlaví/genderu jmenovaných lidí
skloňovanou příponou. Nejsou-li takové názvy doplněny jiným prostředkem
vyjádření pohlaví/genderu, očekává se tím od adresátstva
uvažování a zacházení abstrahující od pohlaví/genderu, jemuž rod pátý napomáhá
poskytnutím takových tvarů dalších slovních druhů,
které nejsou a nemají být stereotypně dávány do souvislosti
s žádným konkrétním pohlavím/genderem (ani jeho případnou absencí).

Pátý rod je vhodný zejmena tehdy, když pohlaví/gender není znám/o
nebo je nejednoznačné/ý (jako v případě smíšené skupiny osob).
Vyjadřuje, že se jedná o osoby, ale současně znamená, že jejich
pohlaví/gender tím zatím není vyjádřen/o a nemá být domýšlen/o.

Je-li rodem pátým označována fyzická osoba, mělu byste o ní uvažovat způsobem,
který nebude pro osobu žádného pohlaví/genderu méně patřičný než pro osoby
jiných pohlaví/genderů, a očekávat pod takovým označením lidskou
(popř. humanoidní) bytost působící jakýmkoliv pohlavním/genderovým dojmem,
přitažlivým i odpudivým, jasným i matoucím, konformním i extravagantním,
vám vlastním i vám cizím,
typickým i zřídkavým, i takovým, jaký jste dosud v životě nezažilu.

\begin{comment}
Co chci sdělit:

X* 5R má být esteticky srovnatelný s rodem ž. a m.ž.
X* je pohlavně/genderově neutrální
X* je náhradou lomítkových tvarů a generického maskulina
X* poskytuje mluvnické prostředky, které nejsou a nemají být stereotypně spojeny se žádným konkrétním pohlavím/genderem
X* nemá vyvolávat dojem určitého pohlaví/genderu
X* odmítá sdělování p/r/g nazývaných osob mluvnickým rodem či přechylovací příponou
X* není-li p/g vyjádřen/o jinými prostředky, požaduje 5R pohl/gend. necitlivost; té napomáhají...
X* 5R je vhodný zejména tehdy, když p/g není znám/o nebo je nejednoznačný/é (např. ve smíšené skupině lidí)

Použití pátého rodu nebrání vyjádření ani popření pohlavnosti/genderovosti
nazývaných osob jinými prostředky, než je mluvnický rod,
jako např. vhodně zvolenými přídavnými jmény, ale není příliš vhodné
pro osoby, které jsou na svoje pohlaví/svůj gender obzvlášť hrdé
a chtějí ho co nejvíc předvádět ostatním.
\end{comment}

\subsection{Vztah s českou mluvnicí}

Toto rozšíření nezavádí žádnou změnu existujících pravidel české mluvnice
ani pravopisu, zavádí však nová slova (zejména podstatná jména a zájmena),
nové slovotvorné postupy k jejich vytváření a nové tvary existujících slov.

Tato nová slova a nové tvary se řídí především mluvnickými pravidly
stanovenými tímto rozšířením. \emph{Ve všech případech, které zde nejsou upraveny,
se však řídí pravidly české mluvnice vyloženými tak, jako by
nová slova a tvary byly tzv.} \textbf{náhradního rodu}\emph{,
což je pro tvary čísla jednotného rod mužský životný
a pro tvary čísla množného rod ženský.}

\ifgmelse{}{Žádný mluvní postředek pátého rodu nevyžaduje použití
generického maskulina.}

\begin{comment}
Jiné slovo (např. vlastní jméno) lze považovat za slovo v pátém rodě
a zacházet s ním tak, jen pokud splňuje všechny následující podmínky:
%
\begin{itemize}
\item označuje osobu (v jednotném čísle musí jít o jednu osobu),
\item je nesklonné (u vícerodého jména stačí nesklonnost v jednom rodě)
\item a nemá jednoznačný rod podle české mluvnice (tzn. buď nemá stanoven žádný rod, jako např. anglická jména; nebo je vícerodé).
\end{itemize}
\end{comment}
