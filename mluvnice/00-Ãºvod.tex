% Pátý rod
% Copyright (c) 2021-2023 Singularis <singularis@volny.cz>
%
% Toto dílo je dílem svobodné kultury; můžete ho šířit a modifikovat pod
% podmínkami licence Creative Commons Attribution-ShareAlike 4.0 International
% vydané neziskovou organizací Creative Commons. Text licence je přiložený
% k tomuto projektu nebo ho můžete najít na webové adrese:
%
% https://creativecommons.org/licenses/by-sa/4.0/
%
\documentclass[10pt,draft]{article}
\usepackage[english,czech]{babel}
\usepackage[xetex,
    layout=a4paper,inner=1.5cm,outer=1cm,top=2cm,bottom=1.5cm,
    twoside,
    layouthoffset=0mm,layoutvoffset=0mm,
    a4paper
    ]{geometry}
\usepackage[xetex,table]{xcolor}  % barevné písmo
\usepackage{changepage}     % prostředí {adjustwidth}
%\usepackage{colortbl}       % barevné pozadí prostředí {tabular}
\usepackage{enumitem}       % nastavení {enumerable} a {itemize}
\usepackage{etoolbox}       % podmíněný překlad apod.
\usepackage{fancyhdr}       % formátování záhlaví a zápatí
\usepackage{fontspec}       % {fontspec} musí být načten až po balíčcích s matematickými symboly!
\usepackage[bottom]{footmisc}% aby se poznámky pod čarou správně umístily
\usepackage{ifthen}         %
\usepackage{longtable}       % vyžadována pro {tabu}
%\usepackage[xetex]{graphicx}% vkládání obrázků
%\usepackage{multicol}       % prostředí {multicols} a {multicols*}
\usepackage{tabu}           % lepší tabulky (prostředí {tabu})
\usepackage{titlesec}       % formátování nadpisů kapitol
%\usepackage[titles]{tocloft}% formátování přehledu („obsahu“)
\usepackage{verbatim}

\newboolean{gm}\setboolean{gm}{true}%
\newcommand{\ifgmelse}[2]{\ifthenelse{\boolean{gm}}{#1}{#2}}

% {fontspec}:
    \setmainfont{Latin Modern Roman}\relax%
    \setsansfont{Latin Modern Sans}\relax%
    \setmonofont{Latin Modern Mono Light}\relax%

    \definecolor{bila}{gray}{1}
    \definecolor{cerna}{gray}{0}
    \definecolor{svetleseda}{gray}{0.9}
    \definecolor{seda}{gray}{0.5}
\begin{document}\rmfamily%
%
% TITULNÍ STRANA
%
%\pagestyle{empty}%
\begin{center}%
{\Huge\bfseries PÁTÝ ROD\\}\medskip%
Návrh podle: Singularis\\%
Verze: 1.3 (září 2023)
\end{center}%
\bigskip%
%
\section{Úvod (deklarace)}

Pátý rod je umělecký projekt genderově neutrálního rozšíření českého jazyka.
Jeho cílem je přinést funkčnější a jedno\-značnější alternativu generického
maskulina při maximálním zachování krásy, bohatství a funkčnosti jazyka
ve všech jeho stylech a formách a uměle mu doplnit prostředky
pro vyjadřování v hovoru i textu, které bude esteticky
a funkčně srovnatelné s rodem ženským a rodem
mužským životným, ale nebude muset nést žádnou informaci o pohlaví, genderu,
věku, dospělosti či sebepojetí zúčastněných osob, ani výslovnou ani předpokládanou,
ani pozitivní ani negativní.

Pátý rod je nový negenderovaný rod užívaný k označování osob;
touto vlastností doplňuje stávající mluvnické rody češtiny:
dva široce užívané k označování osob, ale více či méně genderované
(mužský životný a ženský) a dva sice negenderované, ale k označování osob
převážně neužívané (mužský neživotný a střední).

Pátý rod nabízí především nové názvy osob (např. „studenx“, „mažorex“, „striptéřix“
či „rodičkax“), které zakončením „-x“ (resp. „-ks“), jež je dosavadní češtině cizí,
ale čitelné, a tedy je vhodným kandidátem k jejímu rozšíření, výslovně odmítají
vyjádření pohlaví/genderu jmenovaných lidí či jiných osob skloňovanou příponou.
Tyto názvy mohou být podle rozhodnutí mluvčí doplněny jiným prostředkem
vyjádření pohlaví/genderu nebo ponechány bez jeho vyjádření,
a vyžadovat tak uvažování o nazývaných osobách, jaké nebude nepatřičné
pro osoby žádného pohlaví/genderu. Označení osoby v pátém rodě
zejména není samo o sobě příznakem nebinárství, nedospělosti
či genderové nekonformity a s ničím s toho nemá být pátý rod stereotypně
spojován.

Za účelem napomáhání genderově neutrálnímu chápání nových názvů osob
poskytuje pátý rod rovněž nové tvary ostatních slovních druhů,
pečlivě vybrané a vyvážené, aby nevyvolávaly konzistentní dojem
určitého pohlaví/genderu. Toho se dosahuje nejčastěji hláskami,
které jsou na daném místě slova zcela nové, byť v mezích českého hláskosloví
(např. „hubeneu“), nebo takovými, které mísí či vyvažují žensky a mužsky znějící
varianty (např. „hy“); k jednomu z genderovaných tvarů se pátý rod přiklání
pouze u tvarů genderovaných velmi slabě, z nichž není odečítání pohlaví/genderu
v současné češtině obvyklé (např. „jež“ či „jmenovavši“).

Nové tvary pátého rodu samy o sobě nemají být dávány do souvislosti
s žádným konkrétním pohlavím/genderem ani jeho absencí,
nestaví se však do cesty vyjádřování či popírání pohlavnosti/genderovosti
nazývaných osob či sebe jinými prostředky, než je rod,
např. vyjmenováním v závorce. Pátý rod naopak takové novátorské vyjadřování
pohlavnosti/genderovosti usnadňuje, protože odstraňuje tradiční
zažitá genderová vodítka, která by se s ním dostávala do konfliktu.

Názvy osob rodu pátého jsou vhodné především tehdy,
když pohlaví/gender není znám/o nebo je nejednoznačné/ý
(jako v případě smíšené skupiny osob), ale použitelné jsou ve všech kontextech,
kde by byly použitelné stávající genderově specifické názvy,
a mohou být používány k mnoha různým účelům odpovídajícím deklarovanému
významu pátého rodu.

Pakliže neznáte pohlaví/gender nazývané osoby, protože je označována
či o sobě mluví v pátém rodě či jiným genderově neutrálním způsobem,
mělu byste o ní uvažovat způsobem, který nebude nepatřičný
pro osobu žádného pohlaví/genderu, a očekávat pod takovým označením lidskou
(popř. humanoidní) bytost působící jakýmkoliv pohlavním/genderovým dojmem,
přitažlivým i odpudivým,
jasným i matoucím, konformním i extravagantním,
vám vlastním i vám cizím, typickým i zřídkavým,
i takovým, jaký jste dosud v životě nezažilu.

\begin{comment}
Co chci sdělit:

X* 5R má být esteticky srovnatelný s rodem ž. a m.ž.
X* je pohlavně/genderově neutrální
X* je náhradou lomítkových tvarů a generického maskulina
X* poskytuje mluvnické prostředky, které nejsou a nemají být stereotypně spojeny se žádným konkrétním pohlavím/genderem
X* nemá vyvolávat dojem určitého pohlaví/genderu
X* odmítá sdělování p/r/g nazývaných osob mluvnickým rodem či přechylovací příponou
X* není-li p/g vyjádřen/o jinými prostředky, požaduje 5R pohl/gend. necitlivost; té napomáhají...
X* 5R je vhodný zejména tehdy, když p/g není znám/o nebo je nejednoznačný/é (např. ve smíšené skupině lidí)

Použití pátého rodu nebrání vyjádření ani popření pohlavnosti/genderovosti
nazývaných osob jinými prostředky, než je mluvnický rod,
jako např. vhodně zvolenými přídavnými jmény, ale není příliš vhodné
pro osoby, které jsou na svoje pohlaví/svůj gender obzvlášť hrdé
a chtějí ho co nejvíc předvádět ostatním.
\end{comment}

\subsection{Vztah s českou mluvnicí}

Pro použití pátého rodu není nutná žádná změna existujících pravidel
české mluvnice ani pravopisu, pátý rod však zavádí nová slova (zejména podstatná jména
a zájmena), nové slovotvorné postupy k jejich vytváření a nové tvary
existujících slov.

Tato nová slova a nové tvary se řídí především pravidly stanovenými tímto
návrhem. Ve \emph{všech případech, které zde nejsou upraveny},
se však řídí pravidly české mluvnice vyloženými tak, jako by
nová slova a tvary byly:
\begin{itemize}
\item rodu mužského životného, jde-li o slova v čísle jednotném;
\item rodu ženského, jde-li o slova v čísle množném.
\end{itemize}

Toto dodatečné pravidlo se v této verzi pátého rodu uplatňuje nejčastěji
v případě přivlastňovacích zájmen můj, svůj apod.: „můj studenx“
(jako by „studenx“ bylo rodu mužského životného),
ale „moje studenxe“ (jako by „studenxe“ bylo rodu ženského).

\ifgmelse{}{Žádný mluvní prostředek pátého rodu nevyžaduje použití
generického maskulina.}

\begin{comment}
Jiné slovo (např. vlastní jméno) lze považovat za slovo v pátém rodě
a zacházet s ním tak, jen pokud splňuje všechny následující podmínky:
%
\begin{itemize}
\item označuje osobu (v jednotném čísle musí jít o jednu osobu),
\item je nesklonné (u vícerodého jména stačí nesklonnost v jednom rodě)
\item a nemá jednoznačný rod podle české mluvnice (tzn. buď nemá stanoven žádný rod, jako např. anglická jména; nebo je vícerodé).
\end{itemize}
\end{comment}

\subsection{Zakončení -x, resp. -ks}

Zakončení obecných podstatných jmen pátého rodu lze psát -x nebo -ks bez rozdílu
významu, tzn. např. „studenx“ („bez studenxe“) a „studenks“ („bez studenkse“)
jsou totéž slovo, jen jinak zapsané. Toto neplatí pro vlastní jména
v pátém rodě, u nich je třeba dodržet konkrétní pravopisný zápis.

Tvary zájmena pátého rodu „xdo“ a zájmen od něj odvozených lze psát
s x nebo foneticky (gzdo, bez ksoho, ...).

V tomto dokumentu jsou upřednostňovány tvary s písmenem x.
