% Pátý rod
% Copyright (c) 2021 Singularis <singularis@volny.cz>
%
% Toto dílo je dílem svobodné kultury; můžete ho šířit a modifikovat pod
% podmínkami licence Creative Commons Attribution-ShareAlike 4.0 International
% vydané neziskovou organizací Creative Commons. Text licence je přiložený
% k tomuto projektu nebo ho můžete najít na webové adrese:
%
% https://creativecommons.org/licenses/by-sa/4.0/
%
\section{Slovesa}

Pro pátý rod se používají tvary přechodníků společné rodu ženskému a střednímu, např. „Zaklonivši hlavu, studenx vzhlédlu ke stropu. Dávajíc si pozor na učitexi, nenápadně nahlédlu do skrytého taháku.“

V příčestí minulém i příčestí trpném se používá v jednotném čísle koncovka „-u“: „bylu jsem trpěnu, neslu (jsem), peklu (jsem), mazalu (jsem), ...“ a v množném čísle koncovka „-e“: „byle jsme trpěne, nesle jsme, pekle jsme, mazale jsme“.

Nepravidelné sloveso „jít“ má v minulém čase v pátém rodě vedle systematických tvarů „šlu“ a „šle“ také nepravidelné tvary „šelu“ (v čísle jednotném) a „šele“ (v čísle množném); volba záleží na estetických preferencích mluvčí. Toto se projevuje také u všech tvarů od tohoto slovesa odvozených, např. „našelu jsem, přišelu jsem, odešele jsme, vyšele jsme, sešele jsme se“ apod. Tvary ostatní rodů a čísel („šel, šla, šlo, šli, šly“) a přechodníky se nemění.
