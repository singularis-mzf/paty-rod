% Pátý rod
% Copyright (c) 2021 Singularis <singularis@volny.cz>
%
% Toto dílo je dílem svobodné kultury; můžete ho šířit a modifikovat pod
% podmínkami licence Creative Commons Attribution-ShareAlike 4.0 International
% vydané neziskovou organizací Creative Commons. Text licence je přiložený
% k tomuto projektu nebo ho můžete najít na webové adrese:
%
% https://creativecommons.org/licenses/by-sa/4.0/
%

\section{Číslovky}

{
\hyphenpenalty=10000
\begin{longtabu}spread1pt{|X[1,R]|X[2,L]|X[2,L]|X[2,L]|}%
\hline%
%
\textbf{pád}%
    &\textbf{jedna}     &\textbf{dvě}\\\hline\endhead%
%
1.  &jedno/jednau (-x)  &dvě (-xe)\\\hline%
2.  &jednoho (-xe)      &dvou (-xí)\\\hline%
3.  &jednomu (-xi)      &dvěma (-xemi)\\\hline%
4.  &jedno/jedneu (-xi) &dvě (-xe)\\\hline%
6.  &jednom (-xi)       &dvou (-xích)\\\hline%
7.  &jedním (-xem)      &dvěma (-xemi)\\\hline%
\end{longtabu}
}

V jednotném čísle pátého rodu se u číslovky „jedna“ použije tvar pro rod střední,
přičemž v prvním (a pátém) pádě je dovolen také tvar „jednau“ a ve čtvrtém
tvar „jedneu“. Číslovka dvě má jen množné číslo a v rodě pádém má tvary
společné s rodem ženským.

Příklady: \emph{Do třídy vešlu jedno novau studenx. Do třídy vešlu
jednau novau studenx. Každá kniha je pro jedno noveu studenxi.
Každá kniha je pro jedneu noveu studenxi.}

Některé číslovky (např. řadové) se skloňují podle vzorů přídavných jmen.

Ve všech ostatních případech se použije tvar podle náhradního rodu.
