% Pátý rod
% Copyright (c) 2021, 2023 Singularis <singularis@volny.cz>
%
% Toto dílo je dílem svobodné kultury; můžete ho šířit a modifikovat pod
% podmínkami licence Creative Commons Attribution-ShareAlike 4.0 International
% vydané neziskovou organizací Creative Commons. Text licence je přiložený
% k tomuto projektu nebo ho můžete najít na webové adrese:
%
% https://creativecommons.org/licenses/by-sa/4.0/
%

\section{Zájmena}
%
\subsection{Zájmena osobní a osobní vztažná}

V rodě pátém se používá v jednotném čísle osobní zájmeno třetí osoby \textbf{onu}
a v množném čísle \textbf{one}; první pád vztažného osobního zájmena zní
v obou číslech \textbf{jež}. (Jak je uvedeno v tabulce, ve většině tvarů se
při použití po předložce počáteční „j“ mění v „ň“, psané „n“.)

{
\hyphenpenalty=10000
\begin{longtabu}spread1pt{|X[1,R]|X[3,L]|X[3,L]|X[3,L]|X[2,L]|X[2,L]|}%
\hline%
%
\textbf{pád}    &\textbf{onu (bez př.)} &\textbf{onu (s~př.)}  &\textbf{jež (č.j.)}  &\textbf{one} &\textbf{jež (č.mn.)}\\\hline\endhead%
%
1.  &onu        &               &jež        &one    &jež\\\hline%
2.  &jího, (hy) &(bez) ního     &j/níhož    &j/nich &j/nichž\\\hline%
3.  &jimu/jímu, (mí)%
                &(k) nimu/nímu  &j/nimuž, j/nímuž&j/nim  &j/nimž\\\hline%
4.  &jiho, (hy) &(pro) niho     &j/nihož    &je/ně  &jež/něž\\\hline%
6.  &           &(o) ním        &nímž       &nich   &nichž\\\hline%
7.  &jím        &(s) ním        &j/nímž     &j/nimi &j/nimiž\\\hline%
% jehož/jejíž -> ježíš -> ježoš -> jeníž
\end{longtabu}
}

Nepřízvučné jednoslabičné tvary zájmena „onu“ ve druhém, třetím a čtvrtém
pádě jsou v rodě pátém nepravidelné: pro druhý a čtvrtý pád \textbf{hy},
pro pád třetí \textbf{mí}. Tyto tvary nelze použít
na začátku věty, s důrazem ani po předložce a nelze z nich tvořit vztažné zájmeno.

Zájmeno „sám/sama/sami/samy“ má v prvním pádě tvar \textbf{samu}
v jednotném čísle a \textbf{same} v množném čísle;
ostatním pádům se doporučuje vyhýbat (jsou na ústupu i v jiných rodech),
je možno je např. nahradit tvarem zájmena „samau“.

Příklady krátkých a dlouhých tvarů osobního zájmena ve větách
ve 2., 3. a 4. pádě:

\begin{enumerate}[noitemsep]
\item[2.]Není hy tu. Bez ního to nepůjde.
\item[3.]Řekněte mí, že k nimu přijedu.
\item[4.]Musím hy vidět, mám pro niho dárek.
\end{enumerate}

\emph{Pomůcky pro jednotné číslo:} První pád zní „onu“ (vztažné zájmeno „jež“),
šestý a sedmý „ním“ a ve zbylých třech pádech jsou dvojslabičné tvary
složené z jednoslabičných tvarů zájmena rodu ženského (jí/ji)
a rodů mužských (ho/mu), ve třetím pádě se preferuje spíše snáze
vyslovitelné zkrácené „jimu“ před systematickým „jímu“ a toto zkrácení
je umožněno i ve všech odvozených tvarech třetího pádu (správné jsou
ovšem i tvary s nezkráceným -í-).
Jednoslabičné tvary „hy“ a „mí“ vznikají z hlásek „ho/mu“ a „jí/ji“,
ale použitých v opačném pořadí než u dvojslabičných tvarů.

\emph{Pomůcky pro množné číslo:} První pád zní „one“
(vztažné zájmeno „jež“), ostatní pády se shodují ve všech rodech.

\subsection{Zájmena přivlastňovací}

V jednotném čísle třetí osobě pátého rodu přivlastňujeme zájmenem \textbf{ježí}
(dovolen je i alternativní tvar „jení“), skloňovaným podle vzoru přídavných jmen jarní.
S ním souvisí nesklonné vztažné přivlastňovací zájmeno \textbf{jeníž}.

V množném čísle přivlastňujeme zájmenem \textbf{jejich} (společným všem rodům)
a s ním souvisí vztažné přivlastňovací zájmeno \textbf{jejichž}.

Tvary pro zájmena přivlastňující první a druhé osobě (můj/moje,
náš/naše, tvůj/tvoje, váš/vaše) nejsou upraveny, takže se řeší
podle náhradního rodu.

\subsection{Příklady vztažných tvarů ve větách}

\begin{enumerate}[noitemsep]
\item Zde stojí studenx, jež se proslavilu a jeníž popularita roste.
\item Zde stojí studenx, bez níhož bychom se neobešle a bez jeníž pomoci bychom zkrachovale.
\item Zde stojí studenx, k nimuž bychom měle vzhlížet a k jeníž nalezení bylo nutné výběrové řízení.
\item Zde stojí studenx, pro nihož stojí za to pracovat a pro jeníž souhlas jsme udělale vše.
\item[6.]Zde stojí studenx, o nímž se hodně mluví a o jeníž díle kolují dohady.
\item[7.]Zde stojí studenx, s nímž se dobře spolupracuje a s jeníž chováním nejsou žádné problémy.
\end{enumerate}

\subsection{Zájmena ukazovací}

V jednotném čísle se na třetí osobu pátého rodu ukazuje novým zájmenem
\textbf{tan}, jehož skloňování je následující (od něj se odvodí také
zájmena \textbf{tanhle}, \textbf{tanhletan} a \textbf{tanto}):

{
\hyphenpenalty=10000
\begin{longtabu}spread1pt{|X[1,R]|X[2,L]|X[2,L]|X[2,L]|X[2,L]|}%
\hline%
%
\textbf{pád}%
    &\textbf{tan}&\textbf{tanhle}&\textbf{tanto}&\textbf{tanhletan}\\\hline\endhead%
%
1.  &tan (-x)   &tanhle &tanto  &tanhletan\\\hline%
2.  &tí (-xe)   &tíhle  &títo   &tíhletí\\\hline%
3.  &tí (-xi)   &tíhle  &títo   &tíhletí\\\hline%
4.  &tu (-xi)   &tuhle  &tuto   &tuhletu\\\hline%
6.  &tí (-xi)   &tíhle  &títo   &tíhletí\\\hline%
7.  &tům (-xem)  &tůmhle  &tůmto    &tůmhletům\\\hline%
\end{longtabu}
}

V množném čísle se použije ukazovací zájmeno \textbf{ty},
shodné s rodem mužským neživotným a rodem ženským
(v ho\-vo\-ro\-vém jazyce používaném i pro rod střední).

\subsection{Zájmena tázací, neurčitá a záporná}

Vzhledem k záměru pohlavně/genderově neutrálního vyjadřování nabízí rod pátý
jako synonymní alternativu k tázacímu zájmenu rodu mužského životného „kdo“
nové tázací zájmeno \textbf{xdo} (lze psát foneticky \textbf{gzdo}).
Současně s tím vzniká odpovídající vztažné zájmeno xdo (gzdo),
takže „xdo“ lze použít místo „kdo“ ve všech situacích (vztažné zájmeno
však na rozdíl od tázacího nemá vlastní mluvnický rod,
ale přenáší rod toho, k čemu se vztahuje).

{
\hyphenpenalty=10000
\begin{longtabu}{|X[1,R]|X[1,L]|X[1,L]|X[12,L]|}
\hline%
\textbf{pád}&\textbf{xdo}&\textbf{gzdo}&\textbf{příklad}\\\hline\endhead%
1.  &xdo    &\emph{gzdo}    &Něxdo neznámau zaklepalu. Xdo to bylu? Nebylu to nixdo, to se mi jen zdálo.\\\hline%
2.  &xoho   &\emph{ksoho}   &Projekt se neobejde bez něxoho zodpovědneu.\\\hline%
3.  &xomu   &\emph{ksomu}   &Xomu oblíbeneu byste chtěle dát dárek? Něxomu zdejší?\\\hline%
4.  &xoho   &\emph{ksoho}   &Xoho mladeu máte ráde? Něxoho populární v zahraničí?\\\hline%
6.  &xom    &\emph{ksom}    &Víte, o xom introvertní je tento příběh? O něxom velmi známeu.\\\hline%
7.  &xým    &\emph{ksým}    &S xým spolehlivům byste chtěl prožít svůj život?\\\hline%
\end{longtabu}
}

\noindent Od zájmena xdo (gzdo) se pak předponou či příponou odvodí
nová neurčitá zájmena: něxdo (něgzdo), kdexdo (kdegzdo),
lecxdo (lecgzdo), xdosi (gzdosi),
xdokoli/v (gzdokoli/v) a máloxdo (málogzdo)
a záporné zájmeno nixdo (nigzdo).

%„Slyšelu jsem o něxom, kdo by rádu studovalu na naší škole.“

Nová zájmena tázací, vztažná, neurčitá a záporná není nutno používat
a je možno je kombinovat se stávajícími zájmeny, nakolik to mluvnická
shoda dovolí. Jediným nedostatkem stávajících zájmen kdo (jen tázací),
nikdo apod. je, že ve větách mluvnickou shodou vynutí rod mužský životný.

Tato zájmena existují jen v jednotném čísle,
které zde ovšem nemá význam; zájmeny „kdo“ a „xdo“ (gzdo) se ptáme
v jednotném čísle i na skupiny více osob, např. „Xdo včera vyhrálu
v soutěži dvojic?“

\subsection{Zájmena všeobecná}

Zájmenu všechen/všecek přibývají nové tvary pro rod pátý:

{
%\hyphenpenalty=10000
\newcommand*{\malym}[1]{{\itshape\small(#1)}}%
\begin{longtabu}spread1pt{|X[1,R]|X[2,C]|X[3,L]|X[3,L]|}
\hline%
\textbf{pád}&\textbf{(všechen)}&\textbf{č.jedn.}&\textbf{č.mn.}\\\hline\endhead%
1.  &\malym{všechen/všecek} &všechno/všecko (-x) &všechne/všecke (-xe)\\\hline%
2.  &\malym{všeho}          &všího (-xe)    &všech (-xí)\\\hline%
3.  &\malym{všemu}          &všímu (-xi)    &všem (-xím)\\\hline%
4.  &\malym{všeho}          &všechnu (-xi)  &všechny (-xe)\\\hline%
6.  &\malym{všem}           &vším (-xi)     &všech (-xích)\\\hline%
7.  &\malym{vším}           &vším (-xem)    &všemi (-xemi)\\\hline%
\end{longtabu}
}

Množné číslo „všechne“ se hodí i k samostatnému používání, např. ve větách:
„Všechne vědí, že Země je kulatá. V sále už čekale úplně všechne.“

\subsection{Zájmena skloňovaná podle vz. příd. jmen}
%
Zájmena jakau, kterau, některau, samau, takovau a žádnau se skloňují
podle vzoru přídavných jmen mladau. To platí i pro množné číslo,
kde mají 1. pád: jakyje, kteryje, některyje, samyje, takovyje a žádnyje.
