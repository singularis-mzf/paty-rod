% Pátý rod
% Copyright (c) 2021 Singularis <singularis@volny.cz>
%
% Toto dílo je dílem svobodné kultury; můžete ho šířit a modifikovat pod
% podmínkami licence Creative Commons Attribution-ShareAlike 4.0 International
% vydané neziskovou organizací Creative Commons. Text licence je přiložený
% k tomuto projektu nebo ho můžete najít na webové adrese:
%
% https://creativecommons.org/licenses/by-sa/4.0/
%

\section{Přídavná jména}

Tvary vzorů přídavných jmen pro rod pátý jsou tyto:

{
\hyphenpenalty=10000
\begin{longtabu}{|X[1,r]|X[4,l]|X[4,l]|X[4,l]|X[4,l]|X[10,l]|X[10,l]|}
\hline%
{\setlength{\unitlength}{1em}%
\begin{picture}(0,0)\put(-1.5,0){\mbox{\textbf{pád}}}\end{picture}}%
    &\textbf{mladý/á\newline(č.j.)}&\textbf{mladí/é\newline(č.mn.)}&\textbf{jarní\newline(č.j.)}&\textbf{jarní\newline(č.mn.)}%
    &\textbf{matčin/otcův\newline(č.j.)}&\textbf{matčini/otcovi\newline(č.mn.)}\\\hline\endhead%
1.  &mladau     &mladyje&jarní  &jarní  &\small studenxiv (\ifgmelse{partner}{jedinec}/hrob)\newline studenxiva (kolébka)\newline studenxivo (kuře)%
    &\small studenxivi (\ifgmelse{partneři}{jedinci})\newline studenxivy (hroby/kolébky)\newline studenxiva (kuřata)\\\hline
2.  &mladeu     &mladých&jarní  &jarních&\small studenxiva (\ifgmelse{p.}{j.}/hrobu/kuřete)\newline studenxivy (kolébky)%
    &\small studenxivých (hrobů...)\\\hline
3.  &mladeu     &mladým &jarní  &jarním &\small studenxivě (hrobu...)%
    &\small studenxivým (hrobům...)\\\hline
4.  &mladeu     &mladé  &jarní  &jarní  &\small studenxiva (\ifgmelse{partnera}{jedince})\newline studenxiv (hrob)\newline studenxivu (kolébku)\newline studenxivo (kuře)%
    &\small studenxivy (\ifgmelse{partnery}{jedince}/hroby/kolébky)\newline studenxiva (kuřata)\\\hline%
6.  &mladeu     &mladých&jarní  &jarních&\small studenxivě (\ifgmelse{p.}{j.}/h./...)\newline studenxivu (\ifgmelse{p.}{j.}/h.)%
    &\small studenxivých (hrobech...)\\\hline
7.  &mladům     &mladými&jarní  &jarními%
    &\small studenxivým (\ifgmelse{p.}{j.}/h./kuř.)\newline studenxivou (kolébkou)&\small studenxivými (hroby...)\\\hline
\end{longtabu}
}

\noindent%
Pro přídavná jména \emph{přivlastňovací} platí, že je-li přivlastňované jméno
v rodě pátém, použije se v čísle jednotném tvar, jako by bylo v mužském životném
(např. „učitexiv studenx“, „ženin studenx“, „mužův studenx“)
a v čísle množném, jako by bylo v rodě ženském
(„učitexivy studenxe“, ženiny studenxe“, „mužovy studenxe“).

Jmenné tvary přídavných jmen (jako „rád/ráda/rády/rádi“) se v rodě pátém
tvoří v čísle jednotném koncovkou „-u“ („Bylu jsem rádu, že prší.“)
a v čísle množném koncovkou „-e“. („Byle jsme ráde, že prší.“)
\medskip

\emph{Pomůcky pro jednotné číslo:} U vzoru „mladau“ je první pád
„mladau“ (jako „mladá“, kde je „á“ nahrazeno dvojhláskou „au“),
druhý až šestý pád „mladeu“ (jako „mladé“ či „mladou“, kde je koncovka
nahrazena dvojhláskou -eu) a sedmý pád „mladům“ (jako „mladou“,
kde je „ou“ zkráceno na „ů“ a doplněno o „m“ z tvaru „mladým“).
U vzoru jarní je tvar „jarní“ ve všech pádech bez výjimky.
Koncovky vzoru „studenxiv“ odpovídají vzoru „matčin“, kde je „n“ v koncovce
nahrazeno za „v“ z koncovky vzoru „otcův“.

\emph{Pomůcky pro množné číslo:} Množné číslo vzoru „mladau“ je společné
pro rody mužský životný, ženský a pátý, s výjimkou prvního pádu,
kde má pátý rod tvar „mladyje“.
U vzoru jarní jsou tvary společné pro všechny rody ve všech pádech bez výjimky.
Pro vzor „studenxiv“ platí totéž, co v čísle jednotném.
