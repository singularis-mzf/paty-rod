% Pátý rod
% Copyright (c) 2021-2023 Singularis <singularis@volny.cz>
%
% Toto dílo je dílem svobodné kultury; můžete ho šířit a modifikovat pod
% podmínkami licence Creative Commons Attribution-ShareAlike 4.0 International
% vydané neziskovou organizací Creative Commons. Text licence je přiložený
% k tomuto projektu nebo ho můžete najít na webové adrese:
%
% https://creativecommons.org/licenses/by-sa/4.0/
%
\section{E-forma}

Mluvčí, kterým nevyhovuje zakončení „-u“ v jednotném čísle, si mohou zvolit,
že budou používat ve jmenných tvarech přídavných jmen („rádu“)
a příčestí minulém a trpném u sloves („trpělu“, „trpěnu“) místo
závěrečného „-u“ zakončení množného čísla „-e“ a v prvním pádě
zájmena „onu“ tvar „one“. Přitom se nemění ostatní pády zájmena „onu“
a především se \emph{nijak nemění tvary vzoru „mladau“.}

Tento způsob použití pátého rodu se nazývá e-forma a zmenšuje se u něj
rozlišení mezi jednotným a množným číslem. Zaregistrovalu jsem mluvčí,
jímuž připadá nepohodlná, takže doporučuji nenutit ostatní mluvčí
do jejího používání.

Základní formu a e-formu lze kombinovat v jednom textu jen tehdy,
jde-li o promluvy různých osob či postav (např. přímé řeči) nebo jde
o citace pocházející původně z různých textů. Jedno mluvčí by nemělu
mezi e-formou a jejím nepoužitím volně přecházet a už vůbec ne
je kombinovat v jedné větě.

\section{Shoda přísudku s vícenásobným podmětem}

Při shodě přísudku s vícenásobným podmětem se postupuje tak,
že slovo rodu pátého:
\begin{itemize}
\item je-li v čísle jednotném, má vyšší váhu než
slovo rodu mužského neživotného, ale nižší než slovo rodu mužského
životného;
\item je-li v čísle množném, má vyšší váhu než
slovo rodu středního, ale nižší než slovo rodu ženského.
\end{itemize}

\noindent Proto:

\begin{itemize}
\item „Žebrák-muž, studenx a primátorka nesli vědro. Žebráci-muži, studenx a primátorka nesli vědro.“ (Převážil rod mužský životný.)
\item „Studenx a primátorka nesle vědro. Studenx a starostky nesle vědro.“ (Převážil rod pátý.)
\item „Studenxe a primátorka nesly vědro.“ (Převážil rod ženský, protože název v pátém rodě je v množném čísle.)
\item „Studenx a ježí dvojče nesle vědro. Studenxe a jejich dvojčata nesle vědra.“ (V obou případech převážil rod pátý.)
\item „Dvě aktivistaxe, jedno studenx a primátorka nesle vědro.“ (Převážil rod pátý, díky části podmětu, v níž má jednotné číslo.)
\end{itemize}

%\begin{comment}
\section{Pátý rod a vlastní jména}

Použití pátého rodu s vlastními jmény osob je problematické v tom ohledu,
že vlastní jména jsou typicky podstatná jména či zpodstatnělá přídavná jména,
tedy slova s vlastním mluvnickým rodem. Podle tradičního patriarchálního zvyku
se sice v češtině odvozuje mluvnický rod vlastních jmen fyzických osob od pohlaví,
které jim bylo připsáno při narození, a to i navzdory rodu, který by jim
podle zakončení příslušel (např. „ten Motovidlo“), tento tradiční postup však
nejen že se stává poslední dobou zastaralým, ale především v případě pátého rodu
nedává smysl, protože pátý rod je vůči připsaným pohlavím lhostejný.

Ačkoliv pátý rod rovnou měrou označuje všechny osoby, není mluvnicky správné
ho používat v mluvnické shodě se jmény, která v češtině mají jednoznačný rod
jiný než pátý („Miloš Zeman“, „Aneta Langerová“ apod.).
Jeho použití je naopak v pořádku u nepočeštěných jmen anglických
(např. „Virginia Woolf“, „George Bush“), protože ta rod odpovídající
tomu v češtině nemají. Rovněž lze použití pátého rodu považovat za vhodné
u jmen vícerodých či rodem kolísajících (např. Alex či Singularis),
je-li vhodné je neskloňovat nebo končí-li -x či -ks, takže je možné je skloňovat
dle vzoru studenx.

K rodu pátému lze také přejít tam, kde již mluvnická shoda není nutná
a je možná „shoda podle smyslu“, tedy zpravidla v následujícím souvětí, např.:
„Václav Havel udělil státní vyznamenání. Letos vyznamenalu více osob než loni.“
\ifgmelse{(Viz podobný jev v čísle: „Dav demonstroval na náměstí.
Dožadovali se zvýšení mezd.“)}{}
%\end{comment}

\section{Tipy k používání a nepoužívání rodu pátého}
\begin{itemize}
\item Názvy v rodě pátém mohou být použity k jasnému vyjádření (záměrné i nezáměrné) nejistoty ohledně po\-hla\-ví/gen\-deru části členexí skupiny, např. „Na večírku byli dva účastníci-muži, tři účastnixe a čtyři účastnice.“ Uvedená věta znamená, že na večírku bylo devět osob a minimálně dvě z nich byly muži a minimálně čtyři z nich byly ženy, o dalších třech se pohlaví/gender neví. (To je poměrně pravděpodobná situace.)
\item Názvy v rodě pátém mohou být použity k vysvětlení různých jiných
vespolných pravopisných tvarů názvů osob nebo naopak, např. takto:
„Zítra nás navštíví studenx (student:ka) z Prahy. Zítra nás navštíví student:ka (studenx) z Prahy.“%
\item Názvy v rodu pátém označují osoby, proto se jimi nehodí nahrazovat označení neživých či zemřelých objektů jako např. strašák, sněhulák, nebožtík, drak (na provázku) apod., nejsou-li tyto objekty v daném kontextu personifikovány. Takové nahrazení se nehodí ani v případě, kdy takovým názvem ve skutečnosti osobu označujeme, ale jde o označení metaforické, správně je tedy např. „Ta žena je ale strašák!“
\end{itemize}

\begin{comment}
\item Názvy v rodě pátém mohou být použity při označení jedné osoby v souřadném poměru s názvy v jiných rodech, např. „Náš poslední host byl absolvenx ČVUT, překladatelka odborné literatury a laureát Nobelovy ceny za literaturu.“ Z uvedené věty jasně vyplývá, že se jedná o ženu, protože je to vyjářeno přechýleným femininem „překladatelka“. „Laureát“ je generické maskulinum, které sice může svádět k dojmu, že jde o muže, ale označení téže osoby přechýleným femininem tento výklad vylučuje, takže jediný možný výklad je, že jde o ženu a že to maskulinum je generické. Název „absolvenx“ je pak z hlediska genderu osoby zcela neutrální a označení „host“ je vespolné.
\item Názvy v rodě pátém je velmi nežádoucí dávat do vylučovacího poměru s názvy týchž skupin osob nebo jejich genderově vymezených podskupin v jiných rodech. Např. věta „Zítra nás navštíví student/studenx/studentka z Prahy“ o genderu dané osoby sděluje, že je muž/kdokoliv/žena, což dává chybný smysl. Opakované používání takového tvaru může vytvořit škodlivý stereotyp, že názvy osob v pátém rodě neoznačují ženy a muže. Je ovšem v pořádku použit názvy v jiných rodech k vysvětlení názvu v rodě pátém, např.: „Zítra nás navštíví studenx (student/ka/\ldots) z Prahy“ nebo naopak název v rodě pátém jako vysvětlení složeného tvaru, např.: „Zítra nás navštíví student/ka/\ldots{} (studenx) z Prahy.“ V případě opakovaného použití je vhodné tyto dva způsoby střídat, aby nevznikal dojem hierarchie mezi oběma formami vyjádření.
\end{comment}

\section{Výslovnost}
%
Písmeno „x“ se vyslovuje podle běžných pravidel české výslovnosti.
Pro tvary pátého rodu to v praxi znamená, že „x“ vyslovíte vždy „ks“,
s výjimkou případů, kdy je ve slabice, jejíž poslední souhláska je znělá
(b, d, ď, g, h, j, l, m, n, ň, r, v, z, ž) jako ve slově „xdo“;
v těchto případech se vysloví zněle „gz“, proto se „xdo“ čte „gzdo“.

\section{Poznámky a detaily}
\begin{itemize}
\item\ifgmelse{Od slova „komik“ nelze název v pátém rodě utvořit, protože by byl v konfliktu
s existujícím slovem „komix“}{%
Slovo pátého rodu „komix“ nelze utvořit, protože by bylo v konfliktu s existujícím slovem} (a „komex“ by bylo nesrozumitelné a ošklivé);
jako náhradu doporučuji použít slovo „bavix“, které má téměř stejný význam (až na chybějící asociaci s komikou).
\item Tato verze návrhu neposkytuje způsob, jak utvořit v rodě pátém označení osob podle většiny příbuzenských vztahů, zatím můžeme utvořit např. slova „vnukex“, „vdovax“, „švagrex“ a „tchýněx“.
\item Zatím chybí genderově neutrální, ale sexualizované označení pro sexuálně atraktivní osobu.
\item V této verzi návrhu chybí množnost zdrobňování a zhrubňování názvů osob.
\item Chybí jasné genderově neutrální označení osoby, se kterou má mluvčí romantický vztah (ekvivalent anglických they\-friend, joy\-friend, date\-mate apod.). Navrhnuji „romantěnex“, „romantanx“, popř. kratší „romanax“ nebo „romanex“, mohlo by být i „romantex“. (Jsem zvědavau, co se ujme...)
\end{itemize}

\section{Přehled původcixí}

Většinu tvarů a koncovek jsem navrhlu já (Singularis), většinou s výraznou
inspirací z české mluvnice; z dalších zdrojů a původcixí musím zmínit:

\begin{itemize}
\item Zakončení „-e“ navrhl Michal Pitoňák na svém blogu. (červen 2020)\\{}https://pitonak.blog.respekt.cz/myslenka-nebinarniho-nareci/
\item Tvar „mladyje“ pochází z ruštiny.
\item Estetická pravidla koncovek -(i/e)x byla inspirována z finštiny.
\end{itemize}

\section{Licence}

\noindent (c) 2021-2023 Singularis

Tento dokument je dílo svobodné kultury. Můžete ho šířit a modifikovat pod
podmínkami licence Creative Commons Attribution-ShareAlike 4.0 International
vydané neziskovou organizací Creative Commons. Text licence
můžete najít na webové adrese:

https://creativecommons.org/licenses/by-sa/4.0/

\end{document}
