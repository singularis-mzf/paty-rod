% Pátý rod
% Copyright (c) 2021 Singularis <singularis@volny.cz>
%
% Toto dílo je dílem svobodné kultury; můžete ho šířit a modifikovat pod
% podmínkami licence Creative Commons Attribution-ShareAlike 4.0 International
% vydané neziskovou organizací Creative Commons. Text licence je přiložený
% k tomuto projektu nebo ho můžete najít na webové adrese:
%
% https://creativecommons.org/licenses/by-sa/4.0/
%

\section{Podstatná jména}
%
Podstatná jména v pátém rodě jsou výhradně názvy osob a jako takové se
používají, označují tedy především a rovno\-cenně osoby všech typů a druhů,
právnické i fyzické, děti i dospělé, konformní i extravagantní,
lidi i nelidi (např. ve fikci, fantasy světech apod.).
Mohou být použity také v přeneseném smyslu či k označení
personifikovaných objektů, zvířat apod., ale primárně
se takovým označením zvíře rozumí jedině tehdy, je-li osobou.
(Což v realitě nastává velmi zřídka, ve fikci, obzvláště té pro děti,
je to častější.)

Z podstatných jmen pátého rodu lze skloňovat pouze ta, která končí „-x“
(ostatní zůstávají nesklonná); skloňují se podle vzoru \emph{studenx}:

{
%\hyphenpenalty=10000
\begin{longtabu}spread1pt{|X[1,r]|X[1,r]|X[2,l]|X[2,l]|X[5,l]|}
\hline%
\textbf{pád}&\mbox{}&\textbf{č.jednotné}&\textbf{č.množné}&\textbf{jiné vzory (pro srovnání)}\\\hline\endhead%
1.&&studenx&studenxe        &\itshape kost, muž, \ifgmelse{pán,}{} píseň\\\hline%
2.&(bez)&studenxe&studenxí  &\itshape muže, písně, růže\ifgmelse{, soudce}{}\\\hline%
3.&(ke)&studenxi&studenxím  &\itshape kosti, muži, písni, růži\ifgmelse{, soudci}{}\\\hline%
4.&(pro)&studenxi           &studenxe  &\itshape růži\\\hline% (jedn. číslo: původně „studenxe“, pak „studenxu“)
5.&(oslov.:)&studenxi!&studenxe!    &\itshape kosti, muži, písni\\\hline%
6.&(o)&studenxi&studenxích          &\itshape kosti, muži, písni, růži\ifgmelse{, soudci}{}\\\hline%
7.&(se)&studenxem&studenxemi        &\itshape mužem, pánem\ifgmelse{, soudcem}{}\\\hline%
%\textbf{Původní\newline přípona}&\textbf{Nová\newline přípona}&\textbf{Příklady/vzory}&\textbf{Příklad/y v pátém rodě}\\\hline\endhead%
\end{longtabu}
}

Končí-li první pád čísla jednotného -nex a umožní-li to výslovnost,
„e“ mezi „n“ a „x“ může být (podle estetických preferencí mluvčí)
ve všech ostatních pádech a v čísle množném vynecháno;
např. „našinex“, 2. pád může být „našinxe“ i „našinexe“,
3. pád „našinxi“ i „našinexi“ atd.

\emph{Pomůcka pro jednotné číslo:} Skloňovací koncovka je „-i“ ve všech
pádech, kde má tutéž koncovku vzor „růže“ (3., 4. a 6. pád);
v ostatních případech (1., 2., 5. a 7. pád)
se použije stejná koncovka jako u vzoru „muž“.
%Nejjednodušší možnost,
%jak se naučit skloňovací vzor „studenx“, je vyjít ze vzoru \textbf{muž}
%a zapamatovat si odlišný čtvrtý pád „(pro) studenxu“
%(vycházející ze vzorů předseda a žena) a potlačit zde chybějící tvar „mužovi“.
%Složitější možností je vyjít ze vzoru \textbf{píseň},
%kde si kromě odlišného čtvrtého pádu musíme zapamatovat i sedmý „(se) studenxem“
%(vycházející jen ze vzoru muž).

\emph{Pomůcka pro množné číslo:} V množném čísle se vzor studenx
shoduje se vzorem „píseň“. (A také se vzorem růže, až na chybějící
tvar podvzoru „ulic“.)
