% Pátý rod
% Copyright (c) 2021, 2023 Singularis <singularis@volny.cz>
%
% Toto dílo je dílem svobodné kultury; můžete ho šířit a modifikovat pod
% podmínkami licence Creative Commons Attribution-ShareAlike 4.0 International
% vydané neziskovou organizací Creative Commons. Text licence je přiložený
% k tomuto projektu nebo ho můžete najít na webové adrese:
%
% https://creativecommons.org/licenses/by-sa/4.0/
%

\section{Podstatná jména}
%
Podstatná jména v pátém rodě jsou výhradně názvy osob a jako takové se
používají, označují tedy především a rovno\-cenně osoby všech typů a druhů,
právnické i fyzické, děti i dospělé, konformní i extravagantní,
lidi i nelidi (např. ve fikci, fantasy světech apod.).
Mohou být použity také v přeneseném smyslu či k označení
personifikovaných objektů, zvířat apod., ale primárně
se takovým označením zvíře rozumí jedině tehdy, je-li osobou.
(Což v realitě nastává velmi zřídka, ve fikci, obzvláště té pro děti,
je to častější.)

Z podstatných jmen pátého rodu lze skloňovat pouze ta, která končí „-x“,
„-ks“ nebo „-a“, ostatní zůstávají nesklonná. Jména zakončená „-x“ nebo
„-ks“ se skloňují v jednotném i množném čísle dle vzoru \emph{studenx},
jména zakončená „-a“ se skloňují jen v jednotném čísle dle vzoru
\emph{Opera}.

\subsection{Vzor studenx}
%\clearpage%
{
%\hyphenpenalty=10000
\begin{longtabu}spread1pt{|X[1,r]|X[1,r]|X[2,l]|X[2,l]|X[5,l]|}
\hline%
\textbf{pád}&\mbox{}&\textbf{č.jednotné}&\textbf{č.množné}&\textbf{jiné vzory (pro srovnání)}\\\hline\endhead%
1.&&studenx&studenxe        &\itshape kost, muž, \ifgmelse{pán,}{} píseň\\\hline%
2.&(bez)&studenxe&studenxí  &\itshape muže, písně, růže\ifgmelse{, soudce}{}\\\hline%
3.&(ke)&studenxi&studenxím  &\itshape kosti, muži, písni, růži\ifgmelse{, soudci}{}\\\hline%
4.&(pro)&studenxi           &studenxe  &\itshape růži\\\hline% (jedn. číslo: původně „studenxe“, pak „studenxu“)
5.&(oslov.:)&studenxi!&studenxe!    &\itshape kosti, muži, písni\\\hline%
6.&(o)&studenxi&studenxích          &\itshape kosti, muži, písni, růži\ifgmelse{, soudci}{}\\\hline%
7.&(se)&studenxem&studenxemi        &\itshape mužem, pánem\ifgmelse{, soudcem}{}\\\hline%
%\textbf{Původní\newline přípona}&\textbf{Nová\newline přípona}&\textbf{Příklady/vzory}&\textbf{Příklad/y v pátém rodě}\\\hline\endhead%
\end{longtabu}
}

Končí-li první pád čísla jednotného -nex a umožní-li to výslovnost,
„e“ mezi „n“ a „x“ může být (podle estetických preferencí mluvčí)
ve všech ostatních pádech a v čísle množném vynecháno;
např. „našinex“, 2. pád může být „našinxe“ i „našinexe“,
3. pád „našinxi“ i „našinexi“ atd.

\emph{Pomůcka 1 pro jednotné číslo:} 1. pád je základní tvar bez koncovky,
2., 3. a 4. pád mají stejnou koncovku jako vzor \emph{růže},
5., 6. a 7. mají stejnou koncovku jako vzor \emph{muž} (ovšem bez varianty
-ovi).

\emph{Pomůcka 2 pro jednotné číslo:} Skloňovací koncovka je „-i“ ve všech
pádech, kde má tutéž koncovku vzor „růže“ (3., 4. a 6. pád);
v ostatních případech (1., 2., 5. a 7. pád)
se použije stejná koncovka jako u vzoru „muž“.

\emph{Pomůcka 3 pro jednotné číslo:} Můžeme vyjít z některého z podobných
vzorů píseň, růže nebo muž a zapamatovat si jen odlišné koncovky
v některých pádech.

\emph{Pomůcka pro množné číslo:} V množném čísle se vzor studenx
shoduje se vzory „píseň“ a „růže“.

\subsection{Vzor Opera}

Podstatná jména pátého rodu zakončená -a (především vlastní jména osob)
se v čísle jednotném skloňují podle vzoru Opera (Opera je nebinární démon
z anime „Mairimashita! Iruma-kun“). V čísle množném se neskloňují
a zůstávají ve tvaru prvního pádu čísla jednotného.

{
%\hyphenpenalty=10000
\begin{longtabu}spread1pt{|X[1,r]|X[1,r]|X[2,l]|X[2,l]|X[2,l]|}
\hline%
\textbf{pád}&\mbox{}&\textbf{vz. Opera}&\textbf{vz. žena}&\textbf{vz. \ifgmelse{}{„}předseda\ifgmelse{}{“}}\\\hline\endhead%
1.&&Opera&\itshape žena&\itshape předseda\ifgmelse{}{-muž}\\\hline%
2.&(bez)&Opery&\itshape ženy&\itshape předsedy\ifgmelse{}{-muže}\\\hline%
3.&(k)&Opeřevi&\itshape ženě&\itshape předsedovi\ifgmelse{}{-muži}\\\hline%
4.&(pro)&Operu&\itshape ženu&\itshape předsedu\ifgmelse{}{-muže}\\\hline%
5.&(oslov.:)&Opero!&\itshape ženo!&\itshape předsedo\ifgmelse{}{-muži}!\\\hline%
6.&(o)&Opeřevi&\itshape ženě&\itshape předsedovi\ifgmelse{}{-muži}\\\hline%
7.&(se)&Operou&\itshape ženou&\itshape předsedou\ifgmelse{}{-mužem}\\\hline%
\end{longtabu}
}

\emph{Praktická poznámka:} U vzoru Opera lze (kromě třetího a šestého pádu)
očekávat tendenci k intepretaci jako příznaku ženství, což vyplývá
z toho, že „žena“ je výrazně frekventovanější vzor než „předseda“
a ten je zase frekventovanější než „Opera“. V době psaní tohoto textu mi
není jasno, do jaké míry je to problém a co se proti tomu dá dělat.
Uvítám zkušenosti uživatexí.
