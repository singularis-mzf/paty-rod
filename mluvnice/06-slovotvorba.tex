% Pátý rod
% Copyright (c) 2021-2023 Singularis <singularis@volny.cz>
%
% Toto dílo je dílem svobodné kultury; můžete ho šířit a modifikovat pod
% podmínkami licence Creative Commons Attribution-ShareAlike 4.0 International
% vydané neziskovou organizací Creative Commons. Text licence je přiložený
% k tomuto projektu nebo ho můžete najít na webové adrese:
%
% https://creativecommons.org/licenses/by-sa/4.0/
%

\clearpage
\section{Slovotvorba z existujících názvů osob}

Vycházíme-li ze zpodstatnělého přídavného jména\ifgmelse{\mbox{} (např. „hajný“, „komoří“, „rozhodčí“, „vyučující“)}{},
do pátého rodu ho převedeme jednoduše tak, že ho budeme skloňovat podle pravidel
pátého rodu pro přídavná jména (např. hajný/á~$\Rightarrow$~hajnau).

Tvoříme-li název osoby v rodě pátém z podstatného jména v jiném rodě,
obvyklý postup je následující:

\begin{enumerate}
\item Vezmeme základní (zásadně nepřechýlený a pokud možno nezdrobnělý) název osoby,
první pád čísla jednotného.
\item Podle \emph{tabulky slovotvorných přípon a zakončení} najdeme
odpovídající zakončení a nahradíme ho příslušnou slovotvornou příponou rodu pátého.
Odpovídá-li v tabulce víc zakončení, použijeme to nejdelší.

Při nahrazení může být potřeba zohlednit tvrdost/měkkost souhlásek
či celkovou výslovnost vzniklého slova a podle potřeby může dojít
k měkčení souhlásky před příponou.

Vzhledem k tomu, že pro laickou veřejnost nemusí být zřejmý přesný původ
konkrétního názvu či slovotvorná přípona, jejímž připojením byl odvozen,
jsou správně utvořené také názvy, které se neřídí příponou, kterou slovo
ve skutečnosti vzniklo, ale vznikly pouhým nahrazením příslušné skupiny
hlásek na konci slova. Např. „žák“ \ifgmelse{}{(muž)} není ve skutečnosti
odvozeno příponou „-ák“ (od „ž“?), ale slovo „žáx“, které tak můžeme vytvořit,
je praktičtější než systematické „žákex“, proto je zcela v pořádku,
možná dokonce vhodnější.
Variabilita takové slovotvorby v mezích srozumitelnosti a estetiky není na škodu,
protože teprve praxe dokáže rozlišit, která slova se v jazyce uchytí a která ne.
\end{enumerate}
%
\begin{table}
\newcommand*{\cizipripona}[1]{\mbox{}\ensuremath{\bullet}\hspace{0.5em}\mbox{}#1}
\newcommand*{\jinezakonceni}[1]{\mbox{}\ensuremath{\diamond}\hspace{0.5em}\mbox{}#1}
{
\hyphenpenalty=10000
\begin{longtabu}{|X[1,R]|X[1,R]|X[3,L]|X[3,L]|}
\hline%
{\bfseries\small Pův. př.}&{\bfseries\small Nová př.}&\textbf{Příklady/vzory}&\textbf{Příklad/y v rodě pátém}\\\hline\endhead%
\raggedright\footnotesize{}„“\hspace{-0.25em}~(po~souhl.)%
                &-ex/-ix&lidojed\ifgmelse{}{-muž}, listonoš\ifgmelse{}{-muž}, Čech\ifgmelse{}{-muž}&lidojedex, listonošix, Češix\\\hline%
\raggedright\footnotesize{}„“\hspace{-0.25em}~(po~samohl.)%
                &-x&rada\ifgmelse{}{-muž} (Vacátko)&radax\\\hline%
-\textbf{a}     &-ax    &nestyda\ifgmelse{}{-muž}, starosta\ifgmelse{}{-muž}  &nestydax, starostax\\\hline%
-ka             &-kax   &ochmelka\ifgmelse{}{-muž}, černovláska &ochmelkax, černovláskax\\\hline%
-na             &-nax   &běhna&běhnax\small\ (drbna a klepna jsou vespolné)\\\hline% - nutný výzkum: drbna, běhna, klepna;
-ena            &-ex    &švadlena           &švadlex\\\hline%   [ ] uvažovat i o tvaru „švadlenx“, ale aby nepůsobil příliš žensky (chceme co nejkratší tvary) % podle SSČ
\cizipripona{-ita}&-itax  &jezuita\ifgmelse{}{-muž} &jezuitax\\\hline%
\cizipripona{-ista}&-istax &lingvista\ifgmelse{}{-muž}          &lingvistax\\\hline%

-e\textbf{c}\footnotesize{}~(ne po „n“) &-ex/-ix&běžec\ifgmelse{}{-muž}, stařec\ifgmelse{}{-muž}, Estonec\ifgmelse{}{-muž}    &běžix, Estonex\\\hline%
-\textbf{č}              &-čix   &posluchač\ifgmelse{}{-muž}           &posluchačix\\\hline%
%-a\textbf{č}    &-ačix  &posluchač\ifgmelse{}{-muž}  &posluchačix\\\hline%
-áč             &-áčix  &boháč\ifgmelse{}{-muž}, břicháč\ifgmelse{}{-muž} &boháčix\\\hline%

-c\textbf{e}    &-cix   &dárce\ifgmelse{}{-muž}              &dárcix\\\hline%

-o\textbf{ch}   &-ox    &černoch\ifgmelse{}{-muž}            &černox\\\hline%

-á\textbf{k}    &-áx    &divák\ifgmelse{}{-muž}, chytrák\ifgmelse{}{-muž}, Pražák\ifgmelse{}{-muž}, voják\ifgmelse{}{-muž}  &diváx\\\hline%
-ek             &-ex/-ix&mazánek\ifgmelse{}{-muž}, bezzemek\ifgmelse{}{-muž}& mazánex, bezzemix\\\hline%
-oušek          &-oušix &fanoušek\ifgmelse{}{-muž}           &fanoušix\\\hline%
\cizipripona{-ik}&-ix    &chemik\ifgmelse{}{-muž}, politik\ifgmelse{}{-muž}, logik\ifgmelse{}{-muž}&chemix, politix, logix\\\hline%
-ík             &-ix    &stařík\ifgmelse{}{-muž}, uprchlík\ifgmelse{}{-muž}   &stařix\\\hline%
-čík            &-čík   &plavčík\ifgmelse{}{-muž} & plavčix\\\hline%
-ník            &-nix   &podvodník\ifgmelse{}{-muž}, právník\ifgmelse{}{-muž} &právnix\\\hline%

-\textbf{l}     &-lex   &kutil\ifgmelse{}{-muž}, generál\ifgmelse{}{-muž}     &kutilex\\\hline%
\jinezakonceni{-uál}&-uax   &intelektuál\ifgmelse{}{-muž}        &intelektuax\\\hline% [ ] ? duál, intelektuál, vizuál - nejsou přímo z latiny? určitě cizí přípona, ale jaká?
-tel            &-tex   &učitel\ifgmelse{}{-muž}             &učitex\\\hline%

-a\textbf{n}    &-anex  &Američan\ifgmelse{}{-muž}, skokan\ifgmelse{}{-muž}   &Amerikanex/Američanex, skokanex\\\hline% nutno rozlišit Bosňan × Bosňák
-án             &-áx    &dlouhán\ifgmelse{}{-muž}, indián\ifgmelse{}{-muž}    &indiáx\\\hline%    [ ] nesjednotit? sice není potřeba rozlišit, ale bude těžké rozlišovat mezi -an a -án...
\jinezakonceni{-itán}&-itánex &kapitán\ifgmelse{}{-muž}, Neapolitán\ifgmelse{}{-muž}&kapitánex, Neapolitánex\\\hline% Neapolitán? [ ] Problém: dává jiný výsledek než vzor indián! Může být „kapitáx“? Prozkoumat vojenské hodnosti.
-ín             &-ínex  &vojín\ifgmelse{}{-muž}, podivín\ifgmelse{}{-muž} &vojínex, podivínex\\\hline%   nutno rozlišit voj|ák x voj|ín!
-oun            &-oux   &pěstoun\ifgmelse{}{-muž}    &pěstoux\\\hline%
\cizipripona{-ýn}&-ex    &blondýn\ifgmelse{}{-muž}            &blondex\\\hline%

\cizipripona{-e\textbf{r}}&-ex/-ix&partner\ifgmelse{}{-muž}, teenager\ifgmelse{}{-muž}, boxer\ifgmelse{}{-muž}&partnex, teenagix(!), boxix\\\hline%
\cizipripona{-ér}&-éřix  &režisér\ifgmelse{}{-muž}, reportér\ifgmelse{}{-muž}  &režiséřix\\\hline% kvůli estetice (nechceme „režiséx“)
\cizipripona{-or}&-ox    &revizor\ifgmelse{}{-muž}, programátor\ifgmelse{}{-muž}     &revizox, programátox\\\hline%
\jinezakonceni{-our}&-oux   &hubeňour\ifgmelse{}{-muž}           &hubeňoux\\\hline%
\jinezakonceni{-ýr}&-ex/-ýrex&mušketýr\ifgmelse{}{-muž}, pionýr\ifgmelse{}{-muž} &mušketex, pionýrex {\footnotesize\emph{(dle srozumitelnosti)}}\\\hline%

-a\textbf{ř}    &-ařix  &chatař\ifgmelse{}{-muž}, písař\ifgmelse{}{-muž}      &chatařix\\\hline%
-ář             &-ářix  &cukrář\ifgmelse{}{-muž}, návrhář\ifgmelse{}{-muž}    &cukrářix\\\hline%  nutno rozlišit „zednář“ × „zedník“
\cizipripona{-ionář}&-ionářix&milicionář\ifgmelse{}{-muž}         &milicionářix\\\hline%
\cizipripona{-éř}&-éřix  &kancléř\ifgmelse{}{-muž}            &kancléřix\\\hline% kvůli estetice (nechceme „kancléx“)
-íř             &-ířix  &kreslíř\ifgmelse{}{-muž}, uhlíř\ifgmelse{}{-muž}     &kreslířix\\\hline% kvůli většímu odlišení „rytířix“ × „rytex“ (~ rytec/čice) % podle SSČM
-ýř             &-ýřix  &šenkýř\ifgmelse{}{-muž}&šenkýřix\\\hline% podle SSČM

-a\textbf{s}    &-asix  &kliďas\ifgmelse{}{-muž}             &kliďasix\\\hline%
\jinezakonceni{-ous}&-oux   &divous\ifgmelse{}{-muž}             &divoux\\\hline%


-o\textbf{š}    &-ex    &podloš\ifgmelse{}{-muž}             &podlex\\\hline%
-ouš            &-oušix &starouš\ifgmelse{}{-muž}            &staroušix\\\hline%

\cizipripona{-an\textbf{t}}&-anx   &demonstrant\ifgmelse{}{-muž}, pracant\ifgmelse{}{-muž}   &demonstranx\\\hline%
\cizipripona{-át}&-áx    &kandidát\ifgmelse{}{-muž}           &kandidáx\\\hline%
\cizipripona{-ent}&-enx   &student\ifgmelse{}{-muž}            &studenx\\\hline%
\jinezakonceni{-naut}&-naux    &astronaut\ifgmelse{}{-muž}&astronaux\\\hline%
\jinezakonceni{-eut}&-eux     &terapeut\ifgmelse{}{-muž}&terapeux\\\hline%
\jinezakonceni{-out}&-oux   &žrout\ifgmelse{}{-muž}              &žroux\\\hline%     změna oproti verzi 0.2
\jinezakonceni{-i\textbf{v}}&-ivex  &detektiv\ifgmelse{}{-muž}&detektivex\\\hline%
\end{longtabu}}
\begin{center}\bfseries Tabulka slovotvorných přípon a jiných zakončení ($\bullet$ = cizí přípona, $\diamond$ = jiné zakončení)\end{center}
\end{table}
% Pozor na souvislost „prodava|č“ a „posluch|ač“! Obě cesty musejí vést ke stejným tvarům. (Je mezi nimi vůbec rozdíl?)
% Pozor: Švýcar má nulovou příponu (vzniklo od „Švýcary“)

Tvoření slov v rodě pátém od názvů výrazně pohlavně/genderově
specifických \emph{(muž, žena, matka, otec, syn, dcera, ...)}
je sice možné, ale problematické a ve většině případů nepříliš funkční,
výjimkou jsou v této verzi pátého rodu „(zdravotní) sestrax“, „vdovax“, „vnukex“,
„švagrex“ a „tchýněx“.
Při odvozování ostatních je nutno ověřit, zda v praxi fungují jako genderově
neutrální.

Takzvané \emph{vespolné názvy osob} jsou názvy všech mluvnických rodů,
které jsou v praxi hojně a bez ostychu používány
k nazývání osob všech možných pohlaví/genderů a nejsou s nimi svázány
pohlavně/genderově specifické stereotypy. Jejich rod je pouze mluvnický
a nemá pohlavní/genderový význam. Jsou to slova jako
„člověk“, „rodič“, „osoba“, „miláček“, „mazlíček“, „(filmová) postava“,
„oběť“, „duch“, „předek“, „vzor“, „osobnost“, „potomek“, „drbna“, „protějšek“,
„(filmová) hvězda“, „host“, „celebrita“, „sourozenec“, „klepna“, „chudák“,
„idol“, „chudinka“
a také téměř všechny názvy osob rodu středního (dítě, dvojče, strašidlo,
zlobidlo). Vespolné názvy osob stávajících mluvnických rodů mají být používány tak,
jak jsou, a tvořit od nich názvy v rodě pátém je nevhodné.

Zcela nesmyslné je tvořit názvy v rodě pátém od metaforických pojmenování
jako např. „chodící encyklopedie“.

%Pro dvojici „kníže/kněžna“ navrhuji označení „knižetex“.
%U některých přechýlených názvů může být základní tvar velmi
%nezřetelný, v takovém případě by mělo stačit před nahrazováním
%mechanicky odpojit přechylovací příponu. Tímto způsobem vzniknou názvy
%jako „modelex“, „prostitutex“, „mažoretex“, „uklízečix“ apod.

V převzatých názvech, kde před zakončením -ista, -ik, -iv apod. stojí
neměkčená souhláska d, t či n, se tento jev zachová i po nahrazení
příponou rodu pátého, proto jsou správně utvořená slova
„kritix“, „politix“, „detektivex“, „logix“ či „feministax“.


\section{Estetická pravidla pro přípony „-x“, „-ex“ a „-ix“}

Následující pravidla, motivovaná výslovností, jsou převážně zohledněna v překladové tabulce
slovotvorných přípon; může se je však vyplatit znát pro vypořádání se
s některými okrajovými případy:

\begin{enumerate}
\item Předchází-li před „-x“ tvrdá souhláska kromě „n“, vkládá se mezi ně „e“
(lidojed\emph{e}x, učit\emph{e}x). Jako tvrdá se zde počítá i souhláska „g“.
\item Předchází-li před „-x“ měkká souhláska nebo souhláska „s“,
vkládá se tam „i“ (stař\emph{i}x, posluchač\emph{i}x, soudc\emph{i}x).
\item Samotné „-x“ se používá po všech samohláskách a dvojhláskách
(komunistax, indiáx, revizox, astronaux), ale má-li stát po „í“ či „ý“,
doporučuje se tuto samohlásku krátit, je-li to možné.
\item Předchází-li před „-x“ souhláska „n“, vkládá se mezi ně „e“
u většiny slov v prvním pádě jednotného čísla vždy a v ostatních pádech
je možno (ale nikoliv nutno) vložené „e“ vynechat, umožní-li to výslovnost.
Např. „cizinex“, 2. pád „cizinxe“ i „cizinexe“, ale „partnex“, 2. pád
jen „partnexe“, protože „partnxe“ by nebylo pohodlně vyslovitelné.
Výjimkou jsou slova s příponami „-ant“ a „-ent“, kde se „e“ nevkládá vůbec
(demonstranx, konzultanx, studenx, disidenx).
\item Předchází-li před „-x“ obojetná souhláska kromě „s“, vkládá se tam „i“
nebo (méně často) „e“. Volba závisí jednak na tvaru slova, z něhož vycházíme
(chlapex \textasciitilde{} chlapec/chlapčice; chlapix \textasciitilde{} chlapík/chlapice),
a pak také na potřebách výslovnosti, estetiky nebo na potřebě vyhnout se
nežádoucím slovním asociacím.
\end{enumerate}

Tato pravidla se řídí výslovností, nikoliv pravopisem, proto se pro ně
„q“ chová (zpravidla) jako „v“ a „x“ jako „s“ (resp. „z“)
a je třeba si na to dát pozor u cizích slov, kde se výslovnost od zápisu
odchyluje (proto je správně utvořeno např. slovo „teenagix“,
čteno „týnejdžiks“).

Při slovotvorbě může docházet k měkčení souhlásky, např.
\ifgmelse{od slova „Čech“\ }{}je vhodnější místo „Čechex“
utvořit snáze vyslovitelné „Češix“.
Určitým vodítkem může být prozkoumání přechýleného názvu
osoby\ifgmelse{\ (což je v tomto případě „Češka“, nikoliv „Čechka“)}{},
druhého pádu čísla jednotného, případně i odvozených přídavných
jmen.

\section{Slovotvorba odvozováním}

Názvy osob v rodě pátém lze utvářet i přímo od podstatných jmen
(např. „chatax“), přídavných jmen (např. „chytrex“, „krásix“)
a sloves (např. „kreslix“). Tento způsob utváření
dosud nebyl důkladně popsán a prozkoumán a může způsobovat problémy,
protože od slov v rodě pátém nejsou popsány další slovotvorné cesty,
přesto mohou vznikat nové a samostatné názvy v rodě pátém,
které nebudou mít analogické protějšky v jiných rodech.

\section{Výjimky ve slovotvorbě}

Většina názvů osob v češtině vzniká tak, že nejprve vznikne základní
označení v rodě mužském životném, které je nejfrekventovanější a je primárním
nositelem významu, a od něho je následně připojením přechylovací přípony
odvozen název v rodě ženském, který se odkazuje na význam základního označení
a jsou pod něj podle potřeby „odsouváni“ lidé ženského genderu.
Výše popsané mechanismy pro utváření názvů v rodě pátém jsou optimalizovány
pro tento případ.

Některé názvy osob ale fungují jinak. Mohou to být názvy, kde je název
v rodě ženském názvem základním (vdova, kmotra), nejfrekventovanějším
nebo je primárním nositelem významu a společenského statusu
(prostitutka), případně je jediným možným názvem, který nelze funkčně přechýlit
(rodička, sestra).

S těmito jinak fungujícími názvy je vhodné zacházet individuálně a odtržení
pohlavně/genderově zabarvených zakončení pro převod do pátého rodu
provést nesystematicky. Některá takto vzniklá pohlavně/genderově neutrální slova
(výjimky) jsou:

\newenvironment{slovnik}{%
    \newcommand*{\slovo}[3]{\noindent\textbf{##2} ({\footnotesize z~„##1“}) {---~\itshape{}##3}\par}%
    %\medskip\setlength{\parskip}{3pt}%
    \begin{quote}
    }{%
    \par\end{quote}%
}

\begin{slovnik}%
\slovo{kmotr/a}{kmotrax {\mdseries i} kmotrex}{(několik významů)}%
\slovo{mažoretka}{mažorex}{fyzická osoba v barevné uniformě provádějící sportovní cvičení za pochodu se slavnostním průvodem či kapelou, obvykle ve větší skupině}%
\slovo{modelka}{modelex}{xdo pózuje pořizovatexím uměleckých děl nebo předvádí oblečení ve společnosti}%
\slovo{prostitutka}{prostitux}{fyzická osoba, která provozuje prostituci}%
\slovo{rodička}{rodičkax}{člověk rodící dítě ze své dělohy nebo ho čerstvě porodivší}%
\slovo{(zdravotní) sestra}{sestrax}{zdravotnickau pracovnix}%
\slovo{uklízečka}{uklízečix}{fyzická osoba provádějící úklid za mzdu či plat}%
\slovo{vdova}{vdovax}{osoba, jejíž životní partnex zemřelu}%
\end{slovnik}
