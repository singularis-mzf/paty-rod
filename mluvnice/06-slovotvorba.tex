% Pátý rod
% Copyright (c) 2021 Singularis <singularis@volny.cz>
%
% Toto dílo je dílem svobodné kultury; můžete ho šířit a modifikovat pod
% podmínkami licence Creative Commons Attribution-ShareAlike 4.0 International
% vydané neziskovou organizací Creative Commons. Text licence je přiložený
% k tomuto projektu nebo ho můžete najít na webové adrese:
%
% https://creativecommons.org/licenses/by-sa/4.0/
%

\section{Slovotvorba z existujících názvů osob}

Jde-li o zpodstatnělé přídavné jméno\ifgmelse{\mbox{} (např. „hajný“, „komoří“, „rozhodčí“, „vyučující“)}{},
do pátého rodu ho převedete jednoduše tak, že ho budete skloňovat podle pravidel
pátého rodu pro přídavná jména.

Chcete-li vyjít z podstatného jména, postupujte následovně:

\begin{enumerate}
\item Vezměte základní (nepřechýlený a pokud možno nezdrobnělý název osoby,
z něhož chcete vycházet), vezměte první pád čísla jednotného.
\item Podle \emph{tabulky slovotvorných přípon} určete slovotvornou příponu,
pomocí které vznikl, a nahraďte ji odpovídající slovotvornou příponou rodu pátého.
Přitom může být potřeba zohlednit tvrdost/měkkost souhlásek či celkovou výslovnost
vzniklého slova a podle potřeby může dojít k měkčení souhlásky před příponou.
Není vždy nutno se řídit skutečnou etymologií názvu,
spíše je třeba upřednostnit srozumitelnost, praktičnost a snadnou
výslovnost utvořených slov. \ifgmelse{Např. „žák“ sice není ve skutečnosti
odvozeno příponou „-ák“ (od „ž“?), ale slovo „žáx“,
které tak můžeme vytvořit, je praktičtější než „žákex“, a proto je vhodnější.
Variabilita takové slovotvorby není na škodu, protože teprve praxe dokáže
rozlišit, která slova se v jazyce uchytí a která ne.}{}
\end{enumerate}

\begin{table}
%\clearpage%
{
\hyphenpenalty=10000
\begin{longtabu}{|X[1,R]|X[1,R]|X[3,L]|X[3,L]|}
\hline%
{\bfseries\small Pův. př.}&{\bfseries\small Nová př.}&\textbf{Příklady/vzory}&\textbf{Příklad/y v rodě pátém}\\\hline\endhead%
\raggedright\footnotesize{}„“\hspace{-0.25em}~(po~souhl.)%
                &-ex/-ix&lidojed\ifgmelse{}{-muž}, listonoš\ifgmelse{}{-muž}, Čech\ifgmelse{}{-muž}&lidojedex, listonošix, Češix\\\hline%
\raggedright\footnotesize{}„“\hspace{-0.25em}~(po~samohl.)%
                &-x&rada\ifgmelse{}{-muž} (Vacátko)&radax\\\hline%
-\textbf{a}     &-ax    &nestyda\ifgmelse{}{-muž}, starosta\ifgmelse{}{-muž}  &nestydax\\\hline%
-ena            &-ex    &švadlena           &švadlex\\\hline%   [ ] uvažovat i o tvaru „švadlenx“, ale aby nepůsobil příliš žensky (chceme co nejkratší tvary)
-ista           &-istax &komunista\ifgmelse{}{-muž}          &komunistax\\\hline%
-ka             &-kax   &ochmelka\ifgmelse{}{-muž}           &ochmelkax\\\hline%
%
-e\textbf{c}    &-ex/-ix&běžec\ifgmelse{}{-muž}, stařec\ifgmelse{}{-muž}, Estonec\ifgmelse{}{-muž}, chlapec    &běžix, Estonex, chlapex\\\hline%
-inec           &-inex  &našinec\ifgmelse{}{-muž}            &našinex\\\hline%   „našix“ zní mizerně
-ovec           &-ovex  &klausovec\ifgmelse{}{-muž}          &klausovex\\\hline%
%
-a\textbf{č}    &-ačix  &posluchač\ifgmelse{}{-muž}  &posluchačix\\\hline%
-áč             &-áčix  &boháč\ifgmelse{}{-muž}, břicháč\ifgmelse{}{-muž} &boháčix\\\hline%
-č              &-čix   &prodavač\ifgmelse{}{-muž}           &prodavačix\\\hline%
%
-c\textbf{e}    &-cix   &dárce\ifgmelse{}{-muž}              &dárcix\\\hline%
%
-o\textbf{ch}   &-ox    &černoch\ifgmelse{}{-muž}            &černox\\\hline%
%
-á\textbf{k}    &-áx    &divák\ifgmelse{}{-muž}, chytrák\ifgmelse{}{-muž}, Pražák\ifgmelse{}{-muž}, voják\ifgmelse{}{-muž}  &diváx\\\hline%
-ček            &-čix   &{\small(miláček),} dědeček & dědečix\\\hline%
-ik             &-ix    &chemik\ifgmelse{}{-muž}, politik\ifgmelse{}{-muž}, logik\ifgmelse{}{-muž}&chemix, politix, logix\\\hline%
-ík             &-ix    &stařík\ifgmelse{}{-muž}, uprchlík\ifgmelse{}{-muž}   &stařix\\\hline%
-lek            &-lex   &pisálek\ifgmelse{}{-muž}            &pisálex\\\hline%
-ník            &-nix   &podvodník\ifgmelse{}{-muž}, právník\ifgmelse{}{-muž} &právnix\\\hline%
-oušek          &-oušix &fanoušek\ifgmelse{}{-muž}           &fanoušix\\\hline%
%
-\textbf{l}     &-lex   &kutil\ifgmelse{}{-muž}, generál\ifgmelse{}{-muž}     &kutilex\\\hline%
-tel            &-tex   &učitel\ifgmelse{}{-muž}             &učitex\\\hline%
-uál            &-uax   &intelektuál\ifgmelse{}{-muž}        &intelektuax\\\hline%
%
-a\textbf{n}    &-anex  &Američan\ifgmelse{}{-muž}, skokan\ifgmelse{}{-muž}   &Amerikanex/Američanex, skokanex\\\hline% nutno rozlišit Bosňan × Bosňák
-án             &-áx    &dlouhán\ifgmelse{}{-muž}, indián\ifgmelse{}{-muž}    &indiáx\\\hline%    [ ] nesjednotit? sice není potřeba rozlišit, ale bude těžké rozlišovat mezi -an a -án...
-ián            &-iáx   &hrubián\ifgmelse{}{-muž}            &hrubiáx\\\hline%   (následuje -án)
-ín             &-ínex  &vojín\ifgmelse{}{-muž}              &vojínex\\\hline%   nutno rozlišit voj|ák x voj|ín!
-itán           &-itánex &kapitán\ifgmelse{}{-muž}           &kapitánex\\\hline% Neapolitán? [ ] Problém: dává jiný výsledek než vzor indián! Může být „kapitáx“? Prozkoumat vojenské hodnosti.
-oun            &-oux   &blboun\ifgmelse{}{-muž}, pěstoun\ifgmelse{}{-muž}    &pěstoux\\\hline%
-ýn             &-ex    &blondýn\ifgmelse{}{-muž}            &blondex\\\hline%
%
-áto\textbf{r}  &-átox/-atox&programátor\ifgmelse{}{-muž}, reformátor\ifgmelse{}{-muž}&programátox, reformatox {\footnotesize\emph{(dle výslovnosti)}}\\\hline%
-er             &-ex/-ix&partner\ifgmelse{}{-muž}, teenager\ifgmelse{}{-muž}, boxer\ifgmelse{}{-muž}&partnex, teenagix(!), boxix\\\hline%
-ér             &-éřix  &režisér\ifgmelse{}{-muž}, reportér\ifgmelse{}{-muž}  &režiséřix\\\hline% kvůli estetice (nechceme „režiséx“)
-or             &-ox    &revizor\ifgmelse{}{-muž}, major\ifgmelse{}{-muž}     &revizox, majox\\\hline%
-our            &-oux   &hubeňour\ifgmelse{}{-muž}           &hubeňoux\\\hline%
-ýr             &-ex/-ýrex&mušketýr\ifgmelse{}{-muž}, pionýr\ifgmelse{}{-muž} &mušketex, pionýrex {\footnotesize\emph{(dle srozumitelnosti)}}\\\hline%
%
-a\textbf{ř}    &-ařix  &chatař\ifgmelse{}{-muž}, písař\ifgmelse{}{-muž}      &chatařix\\\hline%
-ář             &-ářix  &cukrář\ifgmelse{}{-muž}, návrhář\ifgmelse{}{-muž}    &cukrářix\\\hline%  nutno rozlišit „zednář“ × „zedník“
-éř             &-éřix  &kancléř\ifgmelse{}{-muž}            &kancléřix\\\hline% kvůli estetice (nechceme „kancléx“)
-íř             &-ířix  &kreslíř\ifgmelse{}{-muž}, uhlíř\ifgmelse{}{-muž}     &kreslířix\\\hline% kvůli většímu odlišení „rytířix“ × „rytex“ (~ rytec/čice)
-onář           &-onářix&milicionář\ifgmelse{}{-muž}         &milicionářix\\\hline%
%
-a\textbf{s}    &-asix  &kliďas\ifgmelse{}{-muž}             &kliďasix\\\hline%
-ous            &-oux   &divous\ifgmelse{}{-muž}             &divoux\\\hline%
%
-o\textbf{š}    &-ex    &podloš\ifgmelse{}{-muž}             &podlex\\\hline%
-ouš            &-oušix &teplouš\ifgmelse{}{-muž}            &teploušix\\\hline%
%
-an\textbf{t}   &-anx   &demonstrant\ifgmelse{}{-muž}, pracant\ifgmelse{}{-muž}   &demonstranx\\\hline%
-át             &-áx    &kandidát\ifgmelse{}{-muž}           &kandidáx\\\hline%
-ent            &-enx   &student\ifgmelse{}{-muž}            &studenx\\\hline%
-out            &-oux   &žrout\ifgmelse{}{-muž}              &žroux\\\hline%     změna oproti verzi 0.2
-ut             &-ux    &astronaut\ifgmelse{}{-muž}, terapeut\ifgmelse{}{-muž}&astronaux\\\hline%
%
-i\textbf{v}    &-ivex  &detektiv\ifgmelse{}{-muž}&detektivex\\\hline%
%
%-čí     &-čix   &rozhodčí           & % vynechat - jde o zpodstatnělé přídavné jméno! (stejně jako např. komoří)
%-dlo    &-dlox  &strašidlo          &strašidlox\\\hline% % vynechat - tvoří názvy rodu středního!
\end{longtabu}}
% Pozor na souvislost „prodava|č“ a „posluch|ač“! Obě cesty musejí vést ke stejným tvarům. (Je mezi nimi vůbec rozdíl?)
% Pozor: Švýcar má nulovou příponu (vzniklo od „Švýcary“)
\begin{center}\bfseries Tabulka slovotvorných přípon\end{center}
\end{table}

Tvoření slov v rodě pátém od názvů výrazně pohlavně/genderově
specifických \emph{(muž, žena, matka, otec, syn, dcera, ...)}
je sice možné, ale problematické a pravděpodobně nepříliš funkční,
mohou však fungovat od nich odvozená pojmenování (např. „zdravotní sestrax“).

Takzvané \emph{vespolné názvy osob} jsou názvy všech mluvnických rodů,
které jsou v praxi hojně a bez ostychu používány
k nazývání osob všech možných pohlaví/genderů a nejsou s nimi svázány
pohlavně/genderově specifické stereotypy. Jejich rod je pouze mluvnický
a nemá pohlavní/genderový význam. Jsou to slova jako
„člověk“, „rodič“, „osoba“, „miláček“, „mazlíček“, „filmová postava“,
„oběť“, „duch“, „předek“, „osobnost“, „potomek“, „protějšek“,
„(filmová) hvězda“, „host“, „sourozenec“, „idol“
a také téměř všechny názvy osob rodu středního (dítě, dvojče, strašidlo).
Vespolné názvy osob stávajících mluvnických rodů mají být používány tak,
jak jsou, a tvořit od nich názvy v rodě pátém je nevhodné.

Zcela nesmyslné je tvořit názvy v rodě pátém od metaforických pojmenování
jako např. „chodící encyklopedie“.

%Pro dvojici „kníže/kněžna“ navrhuji označení „knižetex“.
%U některých přechýlených názvů může být základní tvar velmi
%nezřetelný, v takovém případě by mělo stačit před nahrazováním
%mechanicky odpojit přechylovací příponu. Tímto způsobem vzniknou názvy
%jako „modelex“, „prostitutex“, „mažoretex“, „uklízečix“ apod.

V převzatých názvech, kde před zakončením -ista, -ik, -iv apod. stojí
neměkčená souhláska d, t či n, se tento jev zachová i po nahrazení
příponou rodu pátého, proto jsou správně utvořená slova
„kritix“, „politix“, „detektivex“, „logix“ či „feministax“.


\section{Estetická pravidla pro přípony „-x“, „-ex“ a „-ix“}

Následující pravidla, motivovaná výslovností, jsou převážně zohledněna v překladové tabulce
slovotvorných přípon; může se je však vyplatit znát pro vypořádání se
s některými okrajovými případy:

\begin{enumerate}
\item Předchází-li před „-x“ tvrdá souhláska kromě „n“, vkládá se mezi ně „e“
(lidojed\emph{e}x, učit\emph{e}x). Jako tvrdá se zde počítá i souhláska „g“.
\item Předchází-li před „-x“ měkká souhláska nebo souhláska „s“,
vkládá se tam „i“ (stař\emph{i}x, posluchač\emph{i}x, soudc\emph{i}x).
\item Samotné „-x“ se používá po všech samohláskách a dvojhláskách
(komunistax, indiáx, revizox, astronaux), ale má-li stát po „í“ či „ý“,
doporučuje se tuto samohlásku krátit, je-li to možné.
\item Předchází-li před „-x“ souhláska „n“, vkládá se mezi ně „e“
u většiny slov v prvním pádě jednotného čísla vždy a v ostatních pádech
je možno (ale nikoliv nutno) vložené „e“ vynechat, umožní-li to výslovnost.
Např. „cizinex“, 2. pád „cizinxe“ i „cizinexe“, ale „partnex“, 2. pád
jen „partnexe“, protože „partnxe“ by nebylo pohodlně vyslovitelné.
Výjimkou jsou slova s příponami „-ant“ a „-ent“, kde se „e“ nevkládá vůbec
(demonstranx, konzultanx, studenx, disidenx).
\item Předchází-li před „-x“ obojetná souhláska kromě „s“, vkládá se tam „i“
nebo (méně často) „e“. Volba závisí jednak na tvaru slova, z něhož vycházíme
(chlapex \textasciitilde{} chlapec/chlapčice; chlapix \textasciitilde{} chlapík/chlapice),
a pak také na potřebách výslovnosti, estetiky nebo na potřebě vyhnout se
nežádoucím slovním asociacím.
\end{enumerate}

Tato pravidla se řídí výslovností, nikoliv pravopisem, proto se pro ně
„q“ chová (zpravidla) jako „v“ a „x“ jako „s“ (resp. „z“)
a je třeba si na to dát pozor u cizích slov, kde se výslovnost od zápisu
odchyluje (proto je správně utvořeno např. slovo „teenagix“,
čteno „týnejdžiks“).

Při slovotvorbě může docházet k měkčení souhlásky, např.
\ifgmelse{od slova „Čech“\ }{}je vhodnější místo „Čechex“
utvořit snáze vyslovitelné „Češix“.
Určitým vodítkem může být prozkoumání přechýleného názvu
osoby\ifgmelse{\ (což je v tomto případě „Češka“, nikoliv „Čechka“)}{},
druhého pádu čísla jednotného, případně i odvozených přídavných
jmen.

\section{Slovotvorba odvozováním}

Názvy osob v rodě pátém lze utvářet i přímo od podstatných jmen
(např. „chatax“), přídavných jmen (např. „chytrex“, „krásix“)
a sloves (např. „kreslix“). Tento způsob utváření
však dosud nebyl důkladně popsán a prozkoumán a může způsobovat problémy,
protože od slov v rodě pátém nejsou popsány další slovotvorné cesty.

\section{Slovotvorba zvláštní}

Některé přechýlené názvy osob jsou v češtině frekventované,
aniž by byla běžně známá, srozumitelná a funkční jejich základní podoba
nebo protějšek rodu mužského životného. Některé další byly již utvořeny
zdánlivě přechýlené, nejsou považovány za vespolné a příznakovost ženského rodu
upírá tato označení osobám, které nejsou ženy.

S těmito názvy je vhodné zacházet individuálně a odtržení
pohlavně/genderově zabarvených zakončení provést nesystematicky.
Některá takto vzniklá pohlavně/genderově neutrální slova jsou:

\newenvironment{slovnik}{%
    \newcommand*{\slovo}[3]{\noindent\textbf{##2} ({\footnotesize z~„##1“}) {---~\itshape{}##3}\par}%
    %\medskip\setlength{\parskip}{3pt}%
    \begin{quote}
    }{%
    \par\end{quote}%
}

\begin{slovnik}%
\slovo{mažoretka}{mažorex}{fyzická osoba v barevné uniformě provádějící sportovní cvičení za pochodu se slavnostním průvodem či kapelou, obvykle ve větší skupině}%
\slovo{modelka}{modelex}{xdo pózuje pořizovatexím uměleckých děl nebo předvádí oblečení ve společnosti}%
\slovo{prostitutka}{prostitux}{fyzická osoba, která provozuje prostituci}%
\slovo{rodička}{rodičkax}{člověk rodící dítě ze své dělohy nebo ho čerstvě porodivší}%
\slovo{uklízečka}{uklízečix}{fyzická osoba provádějící úklid za mzdu či plat}%
\end{slovnik}
